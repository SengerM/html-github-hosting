%% LyX 2.3.6.1 created this file.  For more info, see http://www.lyx.org/.
%% Do not edit unless you really know what you are doing.
\documentclass[english]{article}
\usepackage[T1]{fontenc}
\usepackage[latin9]{inputenc}
\usepackage[a4paper]{geometry}
\geometry{verbose,tmargin=2cm,bmargin=2cm,lmargin=1cm,rmargin=1cm}
\usepackage{fancyhdr}
\pagestyle{fancy}
\usepackage{color}
\usepackage{babel}
\usepackage{float}
\usepackage{units}
\usepackage{amstext}
\usepackage[unicode=true,pdfusetitle,
 bookmarks=true,bookmarksnumbered=true,bookmarksopen=false,
 breaklinks=false,pdfborder={0 0 1},backref=false,colorlinks=false]
 {hyperref}

\makeatletter
%%%%%%%%%%%%%%%%%%%%%%%%%%%%%% User specified LaTeX commands.
\usepackage{/home/alf/cloud/lib/lyx/my_preamble}

\makeatother

\usepackage{listings}
\lstset{keywordstyle={\color{keyword_color}\ttfamily\bfseries},
commentstyle={\color{comentarios_color}\itshape},
emphstyle={\color{red}},
breaklines=true,
basicstyle={\ttfamily},
stringstyle={\color{cadenas_color}},
identifierstyle={\color{identifier_color}},
backgroundcolor={\color{fondocodigo_color}},
keepspaces=true,
numbers=left,
xleftmargin=2em,
frame=leftline,
rulecolor={\color{black}},
numbersep=5pt,
tabsize=3}
\begin{document}
\input{\string~/cloud/lib/lyx/macros2020.tex}
\title{Long term tests on irradiated LGADs}
\author{Mat�as Senger}
\date{\today}

\maketitle
\tableofcontents{}

\section{Brief status update}

Detailed setup description and commissioning see \href{https://sengerm.github.io/html-github-hosting/210805_long_term_measurements_setup_commisioning/setup_commissioning.html}{here}.
\begin{itemize}
\item The setup started operating in ``test mode'' (i.e. low bias voltages)
on the first days of August 2021, see \ref{Figure: all data measured}.
\item Setup still incomplete, but partially operative.
\begin{itemize}
\item Continuously monitoring standby bias current and voltage of 8 devices
(\ref{Figure: all data measured}).
\item Capable of measuring IV curves regularly (see Figure~6 in \href{https://sengerm.github.io/html-github-hosting/210805_long_term_measurements_setup_commisioning/setup_commissioning.html\#Figure:\%20example\%20of\%20IV\%20measurement}{this link}).
\end{itemize}
\end{itemize}
\begin{figure}[H]
\begin{centering}
\htmltag{tag_name=image}{src=media/plot_with_all_data.svg}{style=max-width: 100\%;}
\par\end{centering}
\caption{All data measured from the first day up to now.\label{Figure: all data measured}}
\end{figure}


\subsection{Working on right now}
\begin{itemize}
\item Reliable characterization of transimpedance of Chubut board. 
\begin{itemize}
\item Issues triggering without reference detector, trigger level changes
results.
\begin{figure}[H]
\begin{centering}
\htmltag{tag_name=image}{src=media/2.svg}{style=max-width: 100\%;}
\par\end{centering}
\caption{Screenshots from the oscilloscope showing the same measurement with
two different trigger levels.}
\end{figure}
\item Could not make coincidences with another detector and the PIN diode,
don't know why.
\end{itemize}
\item Implementation of the electro-mechanical system to move beta source
and reference detector from one device to the other.
\end{itemize}

\section{Some (maybe) interesting things already observed}

\subsection{Discharge at room temperature?}
\begin{itemize}
\item When devices are brought to room temperature there is some kind of
``discharge bias current'', see~\ref{Figure: strange bias current at room temperature}.
\item For device #8 the current is bigger than when it is biased. 
\begin{figure}[H]
\begin{centering}
\htmltag{tag_name=image}{src=media/1.svg}{style=max-width: 100\%;}
\par\end{centering}
\caption{Strange non-zero bias current when devices are brought to $0\protect\UNIT V$
and room temperature after being biased for \textquotedblleft a long
time\textquotedblright . \label{Figure: strange bias current at room temperature}}
\end{figure}
\end{itemize}

\subsection{Device #6 }

Device #6 is ``wafer 10, type 4, fluence $15\TIMESTENTOTHE{15}\UNIT{neq}\CENTI m^{-2}$'',
it has shown strange bias current behavior. The measured data is shown
in \ref{Figure: all data from device num6}.
\begin{itemize}
\item 2021-Aug-5-1:15. Abrupt increase of current while the beta source
was on top of the device.
\item 2021-Aug-18-15:00. After one week I switched off\footnote{This means 1) setting bias voltage to $0\UNIT V$ and 2) going to
room temperature.} and removed the beta source from device #6. Then, bias current much
smaller (makes sense because there is no beta source now?). 
\item 2021-Aug-19-00:00. Bias current starts to increase. It duplicates
its value after about 14 hours. It is also fluctuating too much, see
at 14:20 when the sampling rate was increased.
\item 2021-Aug-19-15:40. Voltage is brought to $0\UNIT V$, temperature
increased and chamber open for some work. After this, at 16:17 the
setup is working again. Bias current is much smaller, even after bringing
bias voltage back to $111\UNIT V$ at 18:05.
\item 2021-Aug-20-10:26. Bias current starts to increase again. At 2021-Aug-21-6:39
it reaches the current compliance value. 
\item 2021-Aug-23-17:00. Setup is switched off once more and after this
the current of #6 was small again, up to now.
\end{itemize}
\begin{figure}[H]
\begin{centering}
\htmltag{tag_name=iframe}{class=plotly}{src=media/device_num6_all_data.html}
\par\end{centering}
\caption{All data measured for device #6 from the commissioning of the setup
up to now. \label{Figure: all data from device num6}}
\end{figure}


\section{Questions I have}
\begin{itemize}
\item What bias voltage should I set for each device?
\item How often to measure IV curve? Measure IV curve at all?
\item How often measure collected charge?
\begin{itemize}
\item At standby voltage?
\item As a function of voltage?
\end{itemize}
\end{itemize}

\end{document}
