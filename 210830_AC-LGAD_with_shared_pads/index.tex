%% LyX 2.3.6.1 created this file.  For more info, see http://www.lyx.org/.
%% Do not edit unless you really know what you are doing.
\documentclass[english]{article}
\usepackage[T1]{fontenc}
\usepackage[latin9]{inputenc}
\usepackage[a4paper]{geometry}
\geometry{verbose,tmargin=2cm,bmargin=2cm,lmargin=1cm,rmargin=1cm}
\usepackage{fancyhdr}
\pagestyle{fancy}
\usepackage{color}
\usepackage{babel}
\usepackage{float}
\usepackage{units}
\usepackage{textcomp}
\usepackage{url}
\usepackage{amsmath}
\usepackage{amssymb}
\usepackage[unicode=true,pdfusetitle,
 bookmarks=true,bookmarksnumbered=true,bookmarksopen=false,
 breaklinks=false,pdfborder={0 0 1},backref=false,colorlinks=false]
 {hyperref}

\makeatletter
%%%%%%%%%%%%%%%%%%%%%%%%%%%%%% User specified LaTeX commands.
\usepackage{/home/alf/cloud/lib/lyx/my_preamble}

\makeatother

\usepackage{listings}
\lstset{keywordstyle={\color{keyword_color}\ttfamily\bfseries},
commentstyle={\color{comentarios_color}\itshape},
emphstyle={\color{red}},
breaklines=true,
basicstyle={\ttfamily},
stringstyle={\color{cadenas_color}},
identifierstyle={\color{identifier_color}},
backgroundcolor={\color{fondocodigo_color}},
keepspaces=true,
numbers=left,
xleftmargin=2em,
frame=leftline,
rulecolor={\color{black}},
numbersep=5pt,
tabsize=3}
\begin{document}
\input{\string~/cloud/lib/lyx/macros2020.tex}
\title{An exploration of pad shape in AC-LGADs}
\author{Mat�as Senger}
\date{\today}
\maketitle
\begin{abstract}
This time I studied what happens with the spacial resolution of an
AC-LGAD when multiple pads are interconnected together producing one
single, big and extended pad. This is potentially interesting because
it may allow to reduce the density of readout channels per unit area
of detector while keeping its performance, simplifying thus the electronics,
reducing the power consumption and also the amount of data flowing
out of the detector. 
\end{abstract}
\tableofcontents{}

\section{Introduction}

Continuing in my work with AC-LGAD devices~\cite{Reference: First application of the empirical likelihood bla bla bla,Reference: FIrst time-space characterization of an AC-LGAD,Reference: spacial resolution of 200 um AC-LGAD}
now I wanted to test if there could be any improvement by changing
the shape of the pads. It turns out that in AC-LGADs there is almost
total freedom on what the shape of the pads is, it is not restricted
to squares but instead they can take any 2D shape. 

The AC-LGAD samples we have available in our lab have nine square
pads arranged in a 3 by 3 matrix. In \ref{Figure: microscope picture of the LGAD}
there is a picture of such a device. To explore how the shape of the
pads can influence the performance of the detector, I decided to do
a quick test interconnecting some of the pads between themselves as
shown in \ref{Figure: microscope picture of the LGAD}. As can be
seen the result is a device with only four channels, instead of nine,
that still cover the whole area of the original nine pads. Now we
can think that each single-colored pair of pads is indeed a single
pad with a non-connected\footnote{Non-connected in the sense of spaces, i.e. in this sense: \url{https://en.wikipedia.org/wiki/Connected_space}.}
shape. Although I did this just because it was easy to do with the
hardware I had available, this could, in principle, be implemented
in a readout chip connected to the AC-LGAD. 

\begin{figure}[H]
\begin{centering}
\htmltag{tag_name=image}{src=media/1.svg}{style=max-width: 100\%;}
\par\end{centering}
\caption{Microscope picture of the AC-LGAD used for this test, also showing
the way the pads were interconnected resulting in a device with less
pads but covering the whole surface.\label{Figure: microscope picture of the LGAD}}
\end{figure}

For the tests presented in this document the used device was the one
labeled \emph{RSD1 W15-A 5,3 3\texttimes 3 200}. 

\section{Results}

In \ref{Figure: collected charge plots} it is shown the collected
charge in each of the four pads, each connected to one channel in
the oscilloscope, as a function of position. It can be seen each of
the individual ``dual-pads'' as was described before, specifically
the CH1 in \ref{Figure: collected charge plots} is the pad~1 from
\ref{Figure: microscope picture of the LGAD}, the CH2 is pad~2,
and so on.

\begin{figure}[H]
\begin{centering}
\htmltag{tag_name=image}{src=media/2.svg}{style=max-width: 100\%;}
\par\end{centering}
\caption{Collected charge, in arbitrary units, measured as a function of position
along the AC-LGAD using the TCT. \label{Figure: collected charge plots}}
\end{figure}

Unfortunately the measurement was affected by an intermittent source
of noise. This can be seen in the presence of vertical noisy and no-noisy
strips in the plots of \ref{Figure: collected charge plots}. This
noise, of course, worsened the results that will be shown later on.
The process of taking this measurement takes several hours/days, thus
each vertical strip represents a different moment in time. It was
not possible to locate the origin of this noise yet, and it is still
affecting new measurements in the lab. 

The procedure I followed was the same as for my other measurements
(see e.g.~\cite{Reference: First application of the empirical likelihood bla bla bla}).
Namely, I produced one training and one testing dataset which I then
used with the MLE algorithm. Results can be seen in \ref{Figure: reconstruction error mean colormap}.
Here we see the average reconstruction error as a function of position.
In each measured $xy$ point $4$ events were recorded for the testing
dataset, so at each point in the plot in \ref{Figure: reconstruction error mean colormap}
the average is of $4$ events. 
\begin{figure}[H]
\begin{centering}
\htmltag{tag_name=iframe}{class=plotly}{src=media/Reconstruction error mean.html}
\par\end{centering}
\caption{Average reconstruction error for the MLE algorithm as a function of
$xy$ position. \label{Figure: reconstruction error mean colormap}}
\end{figure}

In order to study the performance of the detector, I defined a total
of 5 regions numbered from 1 to 4 and one ``main region'', as shown
in \ref{Figure: reconstruction error mean colormap}. The regions
1 to 4 are just squares while the region named ``main region'' is
the difference between the outer square and the inner square, i.e.
the area between the two squares. For each of these regions the reconstruction
error distribution in $x$, in $y$ and in absolute value $\sqrt{x^{2}+y^{2}}$
is shown in \ref{Figure: x error distribution}, \ref{Figure: y error distribution}
and \ref{Figure: combined error distribution} respectively. To ease
visualization and compare different regions it is possible to enable/disable
traces by clicking in the legend of each plot. We see that events
from regions 1 and 2 have better (smaller) $y$ error, given by the
width of each distribution, than events from regions 3 and 4, and
the inverse is true in the $x$ direction. This can be understood
by the position of each region with respect to the pads: regions 1
and 2 are closer to pads in the $y$ direction and far from pads in
the $x$ direction, creating this a steeper gradient of collected
charge in the $y$ direction and an almost flat dependency in the
$x$ direction, as can be seen in the charge color maps of \ref{Figure: collected charge plots}.
The same applies to regions 3 and 4 inverting $x$ and $y$. 

Events in the region called ``main region'' show, obviously, a combined
effect from the other regions. Both the $x$ and $y$ components of
the reconstruction error follow a strange distribution (see \ref{Figure: x error distribution}
and \ref{Figure: y error distribution}) which seems to have two components:
1) a Gaussian like main peak in the middle and 2) tails that extend
further from this Gaussian main peak. A Gaussian distribution was
fitted to the main peak for each coordinate (see the figures) obtaining
from each 
\[
\LBRACE{\begin{aligned} & \sigma_{x\text{ Gaussian fit}}^{\text{main region}}=36.5\MICRO m\\
 & \sigma_{y\text{ Gaussian fit}}^{\text{main region}}=35.2\MICRO m
\end{aligned}
}
\]
 and thus 
\[
\sigma_{\sqrt{x^{2}+y^{2}}\text{ Gaussian fit}}^{\text{main region}}\approx50.7\MICRO m\text{.}
\]
 If, instead, we look at all the data and calculate its standard deviation
we get 
\[
\LBRACE{\begin{aligned} & \sigma_{x\text{ all data}}^{\text{main region}}=78.8\MICRO m\\
 & \sigma_{y\text{ all data}}^{\text{main region}}=76.8\MICRO m
\end{aligned}
}
\]
 and 
\[
\sigma_{\sqrt{x^{2}+y^{2}}\text{ all data}}^{\text{main region}}=110\MICRO m\text{.}
\]
 

\begin{figure}[H]
\begin{centering}
\htmltag{tag_name=iframe}{class=plotly}{src=media/x reconstruction error distribution.html}
\par\end{centering}
\caption{Distribution of the $x$ component of the reconstruction error. The
different regions are defined in \ref{Figure: reconstruction error mean colormap}.
Traces can be enabled-disabled by clicking in the legend.\label{Figure: x error distribution}}
\end{figure}

\begin{figure}[H]
\begin{centering}
\htmltag{tag_name=iframe}{class=plotly}{src=media/y reconstruction error distribution.html}
\par\end{centering}
\caption{Distribution of the $y$ component of the reconstruction error. The
different regions are defined in \ref{Figure: reconstruction error mean colormap}.
Traces can be enabled-disabled by clicking in the legend.\label{Figure: y error distribution}}
\end{figure}

\begin{figure}[H]
\begin{centering}
\htmltag{tag_name=iframe}{class=plotly}{src=media/Reconstruction error distribution.html}
\par\end{centering}
\caption{Distribution of the combined $\sqrt{x^{2}+y^{2}}$ reconstruction
error. The different regions are defined in \ref{Figure: reconstruction error mean colormap}.
Traces can be enabled-disabled by clicking in the legend.\label{Figure: combined error distribution}}
\end{figure}


\section{Discussion}

How can we tell if these results are good or not? In reference \cite{Reference: spacial resolution of 200 um AC-LGAD}
I measured the spacial resolution of the exact same device when it
was connected ``normally''. In that opportunity I obtained a spacial
resolution of 
\[
\LBRACE{\begin{aligned} & \sigma_{x\text{ interpad region}}^{\text{regular connection}}=6.24\MICRO m\\
 & \sigma_{y\text{ interpad region}}^{\text{regular connection}}=5.96\MICRO m
\end{aligned}
}
\]
 and so 
\[
\sigma_{\sqrt{x^{2}+y^{2}}\text{ interpad region}}^{\text{regular connection}}=8.63\MICRO m
\]
 where the ``interpad region'' is the region between the four pads.
Of course in this case we expect a better resolution because the pitch
is $200\MICRO m$ against about twice in the current work. The question
now is how to compare these two results. One possibility is to define
a ``spacial resolution efficiency'' 
\[
\eta\DEF\frac{\sqrt{A}}{\sigma N}
\]
 where $A$ is the area of some region $\mathcal{R}$ (e.g. any of
the regions in \ref{Figure: reconstruction error mean colormap}),
$\sigma$ is the spacial resolution obtained for region $\mathcal{R}$
and $N$ the number of channels. With this definition $\eta$ increases
with the covered area, it is also bigger for smaller $\sigma$ which
is what we want, and it becomes worse as we increase the number of
channels. Using this spacial resolution efficiency we can compare
the two measurements:
\[
\eta=\LBRACE{\begin{aligned} & 0.624 &  & \text{for the current work}\\
 & 2.39 &  & \text{for the normal configuration}
\end{aligned}
}\text{.}
\]
Based in this quantity, the ``normal configuration'' from reference
\cite{Reference: spacial resolution of 200 um AC-LGAD} seems to be
better than the current configuration mixing pads. 

\section{Conclusion}

A crazy interconnection of the pads in an AC-LGAD was measured with
the objective of studying how the spacial resolution degrades while
keeping the same number of channels but covering a bigger area. The
results showed that the spacial resolution was severely affected,
becoming a factor of about $\gtrsim10$ bigger, while the area covered
with this approach increased in a factor of about $8$. The results,
however, may not be accurate due to the presence of an intermittent
noise in the measuring setup, as was mentioned in the text. A further
study on this subject may be worth.
\begin{thebibliography}{1}
\bibitem{Reference: spacial resolution of 200 um AC-LGAD}Spacial
resolution of $200\MICRO m$ pitch AC-LGAD, https://msenger.web.cern.ch/spacial-resolution-of-200-\textmu m-pitch-ac-lgad/.

\bibitem{Reference: FIrst time-space characterization of an AC-LGAD}First
time-space characterization of an AC-LGAD, \url{https://msenger.web.cern.ch/first-time-space-characterization-of-an-ac-lgad/}.

\bibitem{Reference: First application of the empirical likelihood bla bla bla}First
application of the empirical likelihood function to position reconstruction
in AC-LGAD detectors, \url{https://msenger.web.cern.ch/first-application-of-the-empirical-likelihood-function-to-position-reconstruction-in-ac-lgad-detectors/}.
\end{thebibliography}

\end{document}
