%% LyX 2.3.6.1 created this file.  For more info, see http://www.lyx.org/.
%% Do not edit unless you really know what you are doing.
\documentclass[english]{article}
\usepackage[T1]{fontenc}
\usepackage[latin9]{inputenc}
\usepackage[a4paper]{geometry}
\geometry{verbose,tmargin=2cm,bmargin=2cm,lmargin=1cm,rmargin=1cm}
\usepackage{fancyhdr}
\pagestyle{fancy}
\usepackage{color}
\usepackage{babel}
\usepackage{float}
\usepackage{units}
\usepackage{url}
\usepackage{amsmath}
\usepackage{amssymb}
\usepackage{xargs}[2008/03/08]
\usepackage[unicode=true,pdfusetitle,
 bookmarks=true,bookmarksnumbered=true,bookmarksopen=false,
 breaklinks=false,pdfborder={0 0 1},backref=false,colorlinks=false]
 {hyperref}

\makeatletter
%%%%%%%%%%%%%%%%%%%%%%%%%%%%%% User specified LaTeX commands.
\usepackage{/home/alf/cloud/lib/lyx/my_preamble}

\makeatother

\usepackage{listings}
\lstset{keywordstyle={\color{keyword_color}\ttfamily\bfseries},
commentstyle={\color{comentarios_color}\itshape},
emphstyle={\color{red}},
breaklines=true,
basicstyle={\ttfamily},
stringstyle={\color{cadenas_color}},
identifierstyle={\color{identifier_color}},
backgroundcolor={\color{fondocodigo_color}},
keepspaces=true,
numbers=left,
xleftmargin=2em,
frame=leftline,
rulecolor={\color{black}},
numbersep=5pt,
tabsize=3}
\begin{document}
\input{\string~/cloud/lib/lyx/macros2020.tex}

\global\long\def\NEUTRONEQUIVALENT{\UNIT n_{\text{eq}}}%

\title{Commissioning of the long term studies setup}
\author{Mat�as Senger}
\date{August 5, 2021}
\maketitle
\begin{abstract}
In this document I provide an update on the commissioning of setup
for the long term studies of irradiated LGAD detectors at UZH. This
setup is intended to periodically monitor the characteristics and
performance of irradiated LGAD detectors when they are kept biased
for a long period of time. 
\end{abstract}
\tableofcontents{}

\section{Introduction}

In this work a set of irradiated FBK detectors from the UFSD3.2 production~\cite{Reference: libro nuevo sobre UFSD}
are going to be tested\footnote{More specifically, devices of ``type'' 4 and 10 from wafers 4, 10,
18 and 4A, irradiated to fluences of $4\TIMESTENTOTHE{14}\NEUTRONEQUIVALENT\CENTI m^{-2}$,
$8\TIMESTENTOTHE{14}\NEUTRONEQUIVALENT\CENTI m^{-2}$, $15\TIMESTENTOTHE{14}\NEUTRONEQUIVALENT\CENTI m^{-2}$
and $25\TIMESTENTOTHE{14}\NEUTRONEQUIVALENT\CENTI m^{-2}$. }. The tests consist in keeping them under working conditions (i.e.
low temperature and biased) for an extended period of time, and regularly
monitor a number parameters to look for deviations from the expected
performance. For this, a dedicated setup is under implementation at
UZH consisting of several hardware equipment and customized controlling
and analysis software. 

In this document the first report on its commissioning is presented.
First the whole setup is described. Then some preliminary measurements
are presented.

\section{Setup description}

In~\ref{Figure: block diagram of the setup} a block diagram of the
designed setup is shown. The setup was designed to handle up to eight
devices simultaneously. The devices are individually controlled and
monitored, independently from one another. Two CAEN high voltage power
supplies provide the high voltage to bias the LGADs, while at the
same time are used to monitor the bias current and regularly trace
IV curves of each device. An oscilloscope and a PSI DRS4 evaluation
board\cite{Reference: PSI DRS4} are used to measure the signals coming
out from each LGAD when exposed to beta radiation from an Sr-90 source.
A robotic system moves the Sr-90 beta source and MCP-PMT time reference
from one LGAD to the other. All the setup is automatically controlled
by a computer which runs a customized software. A picture of the setup
is shown in~\ref{Figure: picture of the setup}.

\begin{figure}[H]
\begin{centering}
\htmltag{tag_name=image}{src=media/setup_block_diagram.svg}{style=max-width: 100\%;}
\par\end{centering}
\caption{Block diagram of the designed setup. \label{Figure: block diagram of the setup}}
\end{figure}

\begin{figure}[H]
\begin{centering}
\htmltag{tag_name=image}{src=media/picture_of_the_setup.svg}{style=max-width: 100\%;}
\par\end{centering}
\caption{Picture of the setup in its current state. The climate chamber hosts
the readout boards with the LGADs. The two high voltage power supplies
lie on the first shelve (red devices). The DC power supply is placed
in the higher shelve. The oscilloscope and the PC are on the table.
\label{Figure: picture of the setup}}
\end{figure}


\subsection{Readout board}

To provide flexibility, each device is mounted in an individual readout
board which has a built in amplifier. The board, designed and produced
in the context of this project, is based in the Santa Cruz board~\cite{Reference: Twiki of the UCSC board}.
The circuit has some minor modifications and the board a completely
new layout, to obtain a board with smaller dimensions. For details
on this board and its performance please see reference~\cite{Reference: The Chubut board my website}.
Some pictures of this board can be seen in~\ref{Figure: pics of the Chubut board},
and a microscopy picture of a device mounted in a Chubut board is
shown in~\ref{Figure: microscopy picture}. 
\begin{figure}[H]
\begin{centering}
\htmltag{tag_name=image}{src=media/chubut_board_pics.svg}{style=max-width: 100\%; width: 777px;}
\par\end{centering}
\caption{Pictures of the Chubut board~\cite{Reference: The Chubut board my website},
developed in the context of this project. \label{Figure: pics of the Chubut board}}
\end{figure}

\begin{figure}[H]
\begin{centering}
\htmltag{tag_name=image}{src=media/microscope_picture_of_a_device_bonded.svg}{style=max-width: 100\%;}
\par\end{centering}
\caption{Microscopy picture of a device mounted in the Chubut board. \label{Figure: microscopy picture}}
\end{figure}


\subsection{High voltage power supplies}

The bias voltage for the eight devices is provided by two CAEN modules,
each with four outputs, which can be seen in the picture of~\ref{Figure: picture of the setup}
(red apparatuses on the first shelve). One of the modules is model
DT1419ET and the other is a DT1470ET. The modules are controlled through
the network of the institute, optionally via USB. A simple and easy
pure Python package was developed to control the instruments~\footnote{\url{https://github.com/SengerM/CAENpy}.}.

\subsection{Acquisition system}

To acquire the fast signals produced by MIP particles traversing the
LGADs, a LeCroy WaveRunner 640Zi oscilloscope sampling at $40\GIGA s\UNIT s^{-1}$
is used. This oscilloscope has four input channels. To fully automatize
the setup, a PSI DSR4 evaluation system~\cite{Reference: PSI DRS4}
is planned to be used, though the software to control this device
has not yet been implemented. 

\subsection{Timing reference}

A Photonis~PP2365AC MCP-PMT is going to be used as a reference for
fast and precise triggering of the acquisition system. This device
provides a time resolution $\lesssim10\PICO s$~\footnote{Information provided to us by the manufacturer. Se also references
\cite{Reference: MCP as timing reference 1,Reference: MCP as timing reference 2,Reference: MCP as timing reference 3,Reference: MCP as timing reference 4}
which describe the performance of a similar device for timing of minimum
ionizing particles.}. This device will be mounted in a robotic system, currently in design,
to move together with a radioactive Sr-90 source along each of the
eight test LGADs. The block diagram in~\ref{Figure: block diagram of the setup}
illustrates this.

\section{Standby bias voltage and current monitoring}

During the whole test the detectors will be kept under a standby voltage
that can be configured individually for each device. Both the bias
voltage and the bias current will be monitored constantly. In \ref{Figure: standby bias current}
the very first measurements of standby bias current obtained for eight
detectors simultaneously using this setup is presented. 
\begin{figure}[H]
\begin{centering}
\htmltag{tag_name=iframe}{class=plotly}{src=media/Standby bias current.html}
\par\end{centering}
\caption{Very first measurements of the bias current for eight irradiated LGADs
installed inside the climate chamber. The labels in the legend are
codified in an internal numeration to identify the detectors. \label{Figure: standby bias current}}
\end{figure}


\section{IV-curve measurements}

The current-voltage (IV) characteristic of an LGAD detector provides
valuable information. The setup measures the IV curve of each device
periodically, using the values of bias voltage and current provided
by the high voltage power supplies. In \ref{Figure: example of IV measurement}
an IV-curve measurement example is shown (label ``Measurement setup
through CAEN''). The bias voltage and the current are the values
that were feed into the Chubut board. The current corresponds to the
sum of the current of the four pads (3 wire bonded to ground, one
connected to the amplifier input, see~\ref{Figure: microscopy picture})
plus the guard ring, which is wire bonded to ground. The amplifier
input is kept at $\approx0\UNIT V$ DC, while the backside of the
device is where the bias voltage is applied. 

This measurement (\ref{Figure: example of IV measurement}) is compared
against an IV characterization of the same device made with a probe
station before mounting the device in the readout board. In this case,
the guard ring was grounded, one of the pads was kept fixed at $0\UNIT V$
and the bias voltage was applied to the backside through the chuck.
The current shown under the label ``Probe station'' is the current
flowing through the backside. 

\begin{figure}[H]
\begin{centering}
\htmltag{tag_name=iframe}{class=plotly}{src=media/iv_curve_example.html}
\par\end{centering}
\caption{Example of an IV curve measured by the setup using one of the high
voltage power supplies. The measurement is compared with another measurement
of the same device made with a probe station before mounting the device
in the readout board. \label{Figure: example of IV measurement}}
\end{figure}

As seen, the two measurements differ significantly; the current measured
by the CAEN is considerably higher than the current measured by the
probe station, and the current measured in the probe station has an
extra kink at $\sim80-90\UNIT V$. To account for these differences
let's consider the following expressions for the current in each case
\[
\LBRACE{\begin{aligned} & I_{\text{CAEN}}\approx4I_{\text{pad}}+I_{\text{guard ring}}\\
 & I_{\text{probe station}}\approx I_{\text{pad}}+I_{\text{guard ring}}
\end{aligned}
}
\]
 after the connections described for each case. If we focus in the
region where the gain layer depletes the change in the total current
(i.e. the measured current) should be dominated by the variations
in $I_{\text{pad}}$, i.e. 
\[
\EVALUATEDAT{\frac{\DIFERENTIAL I_{\text{guard ring}}}{\DIFERENTIAL V}}{\text{Gain layer depletes}}{}\ll\EVALUATEDAT{\frac{\DIFERENTIAL I_{\text{pad}}}{\DIFERENTIAL V}}{\text{Gain layer depletes}}{}\text{.}
\]
 Following this idea we should observe that 
\[
\EVALUATEDAT{\frac{\DIFERENTIAL I_{\text{CAEN}}}{\DIFERENTIAL V}}{\text{Gain layer depletes}}{}\approx4\EVALUATEDAT{\frac{\DIFERENTIAL I_{\text{probe station}}}{\DIFERENTIAL V}}{\text{Gain layer depletes}}{}\text{.}
\]
 The depletion of the gain layer is happening between $-35\UNIT V$
and $-40\UNIT V$ approximately. In the measurement using the probe
station the increase between $-35\UNIT V$ and $-40\UNIT V$ is of
\[
\EVALUATEDAT{\frac{\DIFERENTIAL I_{\text{probe station}}}{\DIFERENTIAL V}}{\text{Gain layer depletes}}{}\approx135\NANO A-94.3\NANO A=40.7\NANO A\text{.}
\]
 For the measurement with the CAEN the current increase in the same
voltage range is about 
\[
\EVALUATEDAT{\frac{\DIFERENTIAL I_{\text{CAEN}}}{\DIFERENTIAL V}}{\text{Gain layer depletes}}{}\approx350\NANO A-190\NANO A=160\NANO A\text{.}
\]
The ratio of these two currents is 
\[
\frac{160\NANO A}{40.7\NANO A}=3.9
\]
which seem to indicate that the previous reasoning is correct. 

With respect to the difference at $\approx-90\UNIT V$, the second
kink in the measurement with the probe station, not present on the
measurement with the CAEN, may be due to border effects in the pads
left floating. 

\section{Charge collection measurements}

An important figure of the LGAD detectors is the collected charge.
To obtain good timing capability a charge of about $5\text{-}6\FEMTO C$
is required in the output of the detector~\cite{Reference: libro nuevo sobre UFSD}.
The experimental setup is designed to periodically measure the collected
charge of each device using the Sr-90 source of beta particles. The
software for this is still under development, but as an example in~\ref{Figure: example of signal}
a signal produced by a beta particle in a non-irradiated HPK LGAD
device is shown, together with a number of features extracted by the
analysis software. This signal was recorded in a preliminary test
of the current setup. 
\begin{figure}[H]
\begin{centering}
\htmltag{tag_name=iframe}{class=plotly}{src=media/example_signal.html}
\par\end{centering}
\caption{Example of a signal acquired with the setup using the Sr-90 source
and the oscilloscope. This signal was recorded during a preliminary
test using a non-irradiated HPK LGAD, biased with $111\text{ V}$.
A number of features extracted by the analysis software are shown,
in particular the collected charge (in arbitrary units of $\text{V s}$).
\label{Figure: example of signal}}
\end{figure}


\section{Timing measurements}

Perhaps the most relevant measurement is the time resolution of each
detector; after all the detectors will be used for this. To monitor
the time resolution of each detector a precise MCP-PMT reference detector
will be used as trigger, as described previously. This detector was
already ordered to the manufacturer, but it has not yet arrived to
our lab. In the meantime a non-irradiated CNM LGAD detector with a
time resolution of $27\PICO s$ will be used.

\section{Conclusions}

Work is ongoing on the commissioning of the setup for the long term
tests of irradiated LGAD devices. The setup is already partially operational
and monitoring the standby current of eight detectors. There is still
work to do in terms both of hardware and software. It is expected
to have it fully operational in about 2-4 months. 
\begin{thebibliography}{1}
\bibitem{Reference: libro nuevo sobre UFSD}Ferrero, Marco, Roberta
Arcidiacono, Marco Mandurrino, Valentina Sola, and Nicol� Cartiglia.
An Introduction to Ultra-Fast Silicon Detectors: Design, Tests, and
Performances. Boca Raton: CRC Press, 2021. \url{https://doi.org/10.1201/9781003131946}. 

\bibitem{Reference: The Chubut board my website}The Chubut board,
Mat�as Senger. \url{https://msenger.web.cern.ch/the-chubut-board/}.

\bibitem{Reference: Twiki of the UCSC board}UCSC Single Channel.
\url{https://twiki.cern.ch/twiki/bin/view/Main/UcscSingleChannel}.

\bibitem{Reference: PSI DRS4}PSI DRS4 evaluation board, \url{https://www.psi.ch/en/drs/evaluation-board}.

\bibitem{Reference: MCP as timing reference 1}A. Ronzhin, S. Los,
E. Ramberg, M. Spiropulu, A. Apresyan, S. Xie, H. Kim, A. Zatserklyaniy,
Development of a new fast shower maximum detector based on microchannel
plates photomultipliers (MCP-PMT) as an active element, 21 September
2014, Nuclear Instruments and Methods in Physics Research Section
A: Accelerators, Spectrometers, Detectors and Associated Equipment.
\url{https://doi.org/10.1016/j.nima.2014.05.039}.

\bibitem{Reference: MCP as timing reference 2}A. Bornheim, C. Pena,
M. Spiropulu, S. Xie, Z. Zhang, Precision timing detectors with cadmium-telluride
sensor, 21 September 2017, Nuclear Instruments and Methods in Physics
Research Section A: Accelerators, Spectrometers, Detectors and Associated
Equipment. \url{https://doi.org/10.1016/j.nima.2017.04.024}.

\bibitem{Reference: MCP as timing reference 3}Test Beam Studies Of
Silicon Timing for Use in Calorimetry, A. Apresyan, G. Bolla, A. Bornheim,
H. Kim, S. Los, C. Pena, E. Ramberg,A. Ronzhin, M. Spiropulu, and
S. Xie. \url{https://inspirehep.net/files/f6b1cd929d10bb6fe53679fd2f38d3c7}.

\bibitem{Reference: MCP as timing reference 4}A. Ronzhin, M.G. Albrow,
M. Demarteau, S. Los, S. Malik, A. Pronko, E. Ramberg, A. Zatserklyaniy,
Development of a 10ps level time of flight system in the Fermilab
Test Beam Facility, Nuclear Instruments and Methods in Physics Research
Section A: Accelerators, Spectrometers, Detectors and Associated Equipment.
\url{https://doi.org/10.1016/j.nima.2010.08.025}.
\end{thebibliography}

\end{document}
